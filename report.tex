% Options for packages loaded elsewhere
\PassOptionsToPackage{unicode}{hyperref}
\PassOptionsToPackage{hyphens}{url}
\PassOptionsToPackage{dvipsnames,svgnames,x11names}{xcolor}
%
\documentclass[
]{article}

\usepackage{amsmath,amssymb}
\usepackage{iftex}
\ifPDFTeX
  \usepackage[T1]{fontenc}
  \usepackage[utf8]{inputenc}
  \usepackage{textcomp} % provide euro and other symbols
\else % if luatex or xetex
  \usepackage{unicode-math}
  \defaultfontfeatures{Scale=MatchLowercase}
  \defaultfontfeatures[\rmfamily]{Ligatures=TeX,Scale=1}
\fi
\usepackage{lmodern}
\ifPDFTeX\else  
    % xetex/luatex font selection
  \setmainfont[]{Latin Modern Roman}
  \setmathfont[]{Latin Modern Math}
\fi
% Use upquote if available, for straight quotes in verbatim environments
\IfFileExists{upquote.sty}{\usepackage{upquote}}{}
\IfFileExists{microtype.sty}{% use microtype if available
  \usepackage[]{microtype}
  \UseMicrotypeSet[protrusion]{basicmath} % disable protrusion for tt fonts
}{}
\makeatletter
\@ifundefined{KOMAClassName}{% if non-KOMA class
  \IfFileExists{parskip.sty}{%
    \usepackage{parskip}
  }{% else
    \setlength{\parindent}{0pt}
    \setlength{\parskip}{6pt plus 2pt minus 1pt}}
}{% if KOMA class
  \KOMAoptions{parskip=half}}
\makeatother
\usepackage{xcolor}
\setlength{\emergencystretch}{3em} % prevent overfull lines
\setcounter{secnumdepth}{5}
% Make \paragraph and \subparagraph free-standing
\ifx\paragraph\undefined\else
  \let\oldparagraph\paragraph
  \renewcommand{\paragraph}[1]{\oldparagraph{#1}\mbox{}}
\fi
\ifx\subparagraph\undefined\else
  \let\oldsubparagraph\subparagraph
  \renewcommand{\subparagraph}[1]{\oldsubparagraph{#1}\mbox{}}
\fi


\providecommand{\tightlist}{%
  \setlength{\itemsep}{0pt}\setlength{\parskip}{0pt}}\usepackage{longtable,booktabs,array}
\usepackage{calc} % for calculating minipage widths
% Correct order of tables after \paragraph or \subparagraph
\usepackage{etoolbox}
\makeatletter
\patchcmd\longtable{\par}{\if@noskipsec\mbox{}\fi\par}{}{}
\makeatother
% Allow footnotes in longtable head/foot
\IfFileExists{footnotehyper.sty}{\usepackage{footnotehyper}}{\usepackage{footnote}}
\makesavenoteenv{longtable}
\usepackage{graphicx}
\makeatletter
\def\maxwidth{\ifdim\Gin@nat@width>\linewidth\linewidth\else\Gin@nat@width\fi}
\def\maxheight{\ifdim\Gin@nat@height>\textheight\textheight\else\Gin@nat@height\fi}
\makeatother
% Scale images if necessary, so that they will not overflow the page
% margins by default, and it is still possible to overwrite the defaults
% using explicit options in \includegraphics[width, height, ...]{}
\setkeys{Gin}{width=\maxwidth,height=\maxheight,keepaspectratio}
% Set default figure placement to htbp
\makeatletter
\def\fps@figure{htbp}
\makeatother
% definitions for citeproc citations
\NewDocumentCommand\citeproctext{}{}
\NewDocumentCommand\citeproc{mm}{%
  \begingroup\def\citeproctext{#2}\cite{#1}\endgroup}
\makeatletter
 % allow citations to break across lines
 \let\@cite@ofmt\@firstofone
 % avoid brackets around text for \cite:
 \def\@biblabel#1{}
 \def\@cite#1#2{{#1\if@tempswa , #2\fi}}
\makeatother
\newlength{\cslhangindent}
\setlength{\cslhangindent}{1.5em}
\newlength{\csllabelwidth}
\setlength{\csllabelwidth}{3em}
\newenvironment{CSLReferences}[2] % #1 hanging-indent, #2 entry-spacing
 {\begin{list}{}{%
  \setlength{\itemindent}{0pt}
  \setlength{\leftmargin}{0pt}
  \setlength{\parsep}{0pt}
  % turn on hanging indent if param 1 is 1
  \ifodd #1
   \setlength{\leftmargin}{\cslhangindent}
   \setlength{\itemindent}{-1\cslhangindent}
  \fi
  % set entry spacing
  \setlength{\itemsep}{#2\baselineskip}}}
 {\end{list}}
\usepackage{calc}
\newcommand{\CSLBlock}[1]{\hfill\break\parbox[t]{\linewidth}{\strut\ignorespaces#1\strut}}
\newcommand{\CSLLeftMargin}[1]{\parbox[t]{\csllabelwidth}{\strut#1\strut}}
\newcommand{\CSLRightInline}[1]{\parbox[t]{\linewidth - \csllabelwidth}{\strut#1\strut}}
\newcommand{\CSLIndent}[1]{\hspace{\cslhangindent}#1}

\usepackage{booktabs}
\usepackage{longtable}
\usepackage{array}
\usepackage{multirow}
\usepackage{wrapfig}
\usepackage{float}
\usepackage{colortbl}
\usepackage{pdflscape}
\usepackage{tabu}
\usepackage{threeparttable}
\usepackage{threeparttablex}
\usepackage[normalem]{ulem}
\usepackage{makecell}
\usepackage{xcolor}
\usepackage{arxiv}
\usepackage{orcidlink}
\usepackage{amsmath}
\usepackage[T1]{fontenc}
\makeatletter
\@ifpackageloaded{caption}{}{\usepackage{caption}}
\AtBeginDocument{%
\ifdefined\contentsname
  \renewcommand*\contentsname{Table of contents}
\else
  \newcommand\contentsname{Table of contents}
\fi
\ifdefined\listfigurename
  \renewcommand*\listfigurename{List of Figures}
\else
  \newcommand\listfigurename{List of Figures}
\fi
\ifdefined\listtablename
  \renewcommand*\listtablename{List of Tables}
\else
  \newcommand\listtablename{List of Tables}
\fi
\ifdefined\figurename
  \renewcommand*\figurename{Figure}
\else
  \newcommand\figurename{Figure}
\fi
\ifdefined\tablename
  \renewcommand*\tablename{Table}
\else
  \newcommand\tablename{Table}
\fi
}
\@ifpackageloaded{float}{}{\usepackage{float}}
\floatstyle{ruled}
\@ifundefined{c@chapter}{\newfloat{codelisting}{h}{lop}}{\newfloat{codelisting}{h}{lop}[chapter]}
\floatname{codelisting}{Listing}
\newcommand*\listoflistings{\listof{codelisting}{List of Listings}}
\makeatother
\makeatletter
\makeatother
\makeatletter
\@ifpackageloaded{caption}{}{\usepackage{caption}}
\@ifpackageloaded{subcaption}{}{\usepackage{subcaption}}
\makeatother
\ifLuaTeX
  \usepackage{selnolig}  % disable illegal ligatures
\fi
\usepackage{bookmark}

\IfFileExists{xurl.sty}{\usepackage{xurl}}{} % add URL line breaks if available
\urlstyle{same} % disable monospaced font for URLs
\hypersetup{
  pdftitle={Spatial Modeling of Cardiovascular Disease Incidence Positively Associated with PM2.5},
  pdfauthor={Johan Booc; Christina Kim; Shombit Roy},
  pdfkeywords={Fine Particulate Matter (PM2.5), Cardiovascular Disease
(CVD), Cardiovascular Mortality (CVM)},
  colorlinks=true,
  linkcolor={blue},
  filecolor={Maroon},
  citecolor={Blue},
  urlcolor={Blue},
  pdfcreator={LaTeX via pandoc}}

\usepackage{lineno}
\linenumbers
\usepackage{setspace}
\doublespacing
\newcommand{\runninghead}{A Preprint }
\renewcommand{\runninghead}{CVD Related to PM2.5 }
\title{Spatial Modeling of Cardiovascular Disease Incidence Positively
Associated with PM2.5}
\def\asep{\\\\\\ } % default: all authors on same column
\author{\textbf{Johan Booc}\\Department of Statistics\\Texas A\&M
University\\\\\href{mailto:jbooc24@tamu.edu}{jbooc24@tamu.edu}\asep\textbf{Christina
Kim}\\Department of Statistics\\Texas A\&M
University\\\\\href{mailto:christinaykim3@tamu.edu}{christinaykim3@tamu.edu}\asep\textbf{Shombit
Roy}\\Department of Statistics\\Texas A\&M
University\\\\\href{mailto:shombit123@tamu.edu}{shombit123@tamu.edu}}
\date{}
\begin{document}
\maketitle
\begin{abstract}
Cardiovascular disease (CVD) is the leading cause of death in the United
States. By using a spatial modeling technique (geographically weighted
regression) we observed that the concentration of PM 2.5 is centered in
the Stroke Belt region after adjusting for median household income and
unemployment rates (finding similar correlations studied in the past).
Policymakers and health practitioners can use these results to identify
targeted interventions to curb the increasing rates of CVD, and help to
halt one of the world's deadliest diseases.
\end{abstract}
{\bfseries \emph Keywords}
\def\sep{\textbullet\ }
Fine Particulate Matter (PM2.5) \sep Cardiovascular Disease (CVD) \sep 
Cardiovascular Mortality (CVM)


\section{Introduction}\label{sec-intro}

As cardiovascular disease is the leading cause of death in the US,
numerous past studies have been done on this disease, specifically on
the older population the 65+ year age group. Our study aims to further
investigate the impact, focusing on the 18-44-year-old age population,
since generally, there has been less interest in the effects of CVD on
them.~

Our approach involves analyzing this relationship to produce risk
prediction estimates for our specified age group. There are several
factors involved in influencing the incidence rates of CVD from
genetics, lifestyle, diet, and smoking/alcohol habits. We look into the
covariates - PM 2.5 concentration, median household income, and
unemployment rates - to analyze the spatial variation for our specified
age population, thus decreasing future CVD incidence rates.
Specifically, we focus on PM 2.5 concentration as the world continues to
depend on fossil fuels/natural gas and climate change is an ongoing
concern. Earlier studies such as (Warsito et al. 2018) and (Liu et al.
2020)highlight the modern-day threats from air pollution using a Robust
GWR model and Bayesian-temporal model, respectively. From studying our
covariates, we can examine the effects of certain CVD risk factors for
the year 2015.~

The geographically weighted regression model uses local variables and
weights to produce statistical visualizations of the US's regional
variation between our response variable (CVD deaths) and its covariates.
Other methods, such as the traditional linear regression model and
clustering, have been used in past CVD studies. However, we found the
GWR model allows us to see each covariate's local significance and
magnitude. The spatial distribution aspect of the GWR model highlights
regions that are concentrated with CVD rates, whether incidence
percentage rates are increasing or decreasing, and any significant
correlations. This makes the GWR model efficient in analyzing the
dynamic relationship of CVD rates and the factors that contribute to
it.~

Using a geographically weighted regression approach, the overall outcome
of our study aids policymakers and health practitioners in implementing
the necessary interventions for targeted regions that are being
influenced the most by our covariates. For public health experts, it's
ideal to tailor aid to each specific region of the US first to
eventually reach a national decrease in CVD incidence rates.

\section{Related Works}\label{related-works}

There's a growing popularity of using spatial models in the
epidemiological domain to analyze the distribution and factors of a
disease. The techniques often used differ between studies but all have
the same goal: to reduce the disease incidence and mortality rates.
Statistical regression models are popular methods for CVD studies. The
underlying causes for CVD are dynamic and can be seen as an
interconnected web. (Zelko et al. 2023) examines the relationship
between CVD and covariates similar to our study (air pollution, social
determinants, and county-level data). From using a GWR model, their
results found that counties in the South had the highest exposure to PM
2.5 concentrations whereas counties in the Northeast had the lowest. A
strong correlation was also found between household income, race, and
healthcare access. Overall, there are several causes to consider with
CVD, and where the covariates are concentrated is another important
consideration.

Compared to traditional regression models, the geographically weighted
model (GWR) explores spatial data by considering varied coefficients for
a certain spatial unit (Gebreab and Diez Roux 2012). The GWR model
contains the parameter, neighborhood (also known as bandwidth) and
builds on the weighted least squares method to estimate the regression
coefficients. We want to see values closer to the point of interest
since they carry more weight, thus having a greater influence.~

The Ordinary Least Squares (OLS) model is a popular choice of method for
public health studies because it generates the regression coefficients
on a global scale and captures the average differences between
covariates. However, this means spatial variability between units can be
easily hidden for the OLS model. Our paper acknowledges that PM 2.5
concentration and socioeconomic factors are not spatially constant with
cardiovascular disease. The GWR model is more suitable for our study
because the goal of public health is to improve the population as a
whole. To get to the root cause of CVD mortality rates and see future
improvement, the OLS model should be treated as the `null' model, with
the GWR model used to test and verify that it is a statistically
significant better fit.

The GWR model spatially displays a relationship between CVD deaths and
their covariates and analyzes disparities at the local scale. Errors are
also minimalized between the actual model and any estimates. This makes
the GWR model suitable for seeing the socioeconomic factors that affect
different regions and narrows down our focus to the areas in need of
improvement, which helps health practitioners implement policies for
that region. Past studies (Zelko et al. 2023),(Terry et al. 2023),
and(Singh et al. 2019) tend to focus on the trends of socioeconomic
covariates and their spatial patterning. However, our study extends past
studies by including PM 2.5 concentrations as one of our covariates.
From this, we can fully understand the dynamic relationship between
counties and CVD incidence/mortality rates.~

Risk assessment and risk estimates uncover the key factors associated
with CVD. Because the concentration of specific races varies by county,
we studied the dose-response relationship of PM 2.5 concentrations and
the socioeconomic covariates of CVD mortality at the county level. This
helps researchers and health practitioners to develop the necessary
risk-preventative measures and allocate resources to the areas that need
them the most. By putting the focus on the county level (rather than on
individual states/nationally), resources can be allocated accordingly to
the regions that need the most assistance. This can lead to a reduction
in CVD mortality rates, both nationally and globally.

\section{Methods}\label{methods}

\subsection{Data Collection}\label{data-collection}

A data set of Medicare services and claims from the Centers for Disease
Control and Prevention (CDC) website was loaded to analyze Medicare
claims data, specifically Cardiovascular death rates across different
counties in the US. Racial and geographic data were retrieved from the
Census and TIGER Bureau. The median income was extracted from the CENSUS
API. The air quality index was extracted from the NASA PM 2.5
Concentration dataset. Finally, the unemployment rate was extracted from
the CDC website. These data sources were collected and prepared for
analysis to understand the relationship between these factors and
Cardiovascular death rates. The code used to extract and filter the data
is available in our GitHub repository:
\url{https://github.com/jbooc117/STAT489-Project.git}.

\subsection{Data Preprocessing}\label{data-preprocessing}

We took several steps to ensure the reliability and accuracy of the
results when preparing the data for statistical analysis. We integrated
the data from various sources into one file and grouped it by year,
county, and geometries. This integration was achieved through coding in
RStudio Version 4.3.2, specifically focusing on data from 2015, which
allowed for a consistent time frame across all data sets. During the
cleaning and processing phase, we removed features with empty geometries
from the shapefile to ensure the removal of missing values in the
dataset, which led to Nantucket County being removed as it had
incomplete data and was removed from our analysis. Also, we removed
Alaska and Hawaii in this dataset as they are geographically separate
from the US. Centroids of the multi-polygon geometries were calculated
to provide a single point representing the location of these complex
shapes.~

\subsection{Statistical Analysis}\label{statistical-analysis}

A geographically weighted regression (GWR) model was used in the
statistical analysis, and it was run using the GWmodel library (Gollini
et al. 2015). The functions within the GWmodel library were integral in
setting up the model framework and executing the analysis, providing
tools for spatial data manipulation, regression modeling, and data
visualization. The optimal bandwidth for the GWR model was estimated
using cross-validation with hyperparameters we chose to use, such as
Gaussian kernel and fixed bandwidth. The reason we chose the Gaussian
Kernel was its smoothness. We set adaptiveness to false because the goal
is to systematically compare coefficients across different geographic
regions. A fixed bandwidth can help ensure that each region is analyzed
under the same spatial constraints, and its cheaper. This function
modeled the response variable, Cardiovascular death rate, per 100,000
county residents:

\[
y = \beta_0*\%White +\beta_1* \%Black + \beta_2 *\%Hispanic + \beta_3 *\%Asian + \beta_4 *PM2.5 + \beta_5 * Median Income + \beta_6 * \%Unemployed
\]

To calculate t-values from the GWR model, we introduced a correction for
the t-values. This ensures accurate significance testing as the t-values
from the GWR model do not follow a regular t-distribution. Then, the
cumulative distribution function of the adjusted T-distribution is used
to calculate the p-values. We conducted a two-tailed test to find
significant differences, and effective degrees of freedom were found
from the GWR diagnostic output.

\subsection{Mapping}\label{sec-mapping}

We generated geographic plots to visualize the spatial distribution of
the dependent variable (deaths from CVD per 100,000 people) across
different counties in the US, highlighting the significance of the
variables, and plotted the significance of each variable within
different regions to assess the overall effect on the CVD death rate.

This approach allows for examining how various socioeconomic,
environmental, and demographic factors influence health outcomes across
different regions in the United States. We showed this in our simulated
data where we used a piecewise function on the coordinates, which
divides the geographical space into four quadrants and assigns different
coefficients to them to model for spatial heterogeneity, which can be
viewed in Figure~\ref{fig-A1}. Also, a spatial autocorrelation test
(Moran's I) was performed on the residuals of the GWR model. Based on
the table and graph in our appendix, we have a high standard deviation,
low p-value, and a moderate Moran's I, suggesting that the residuals of
the GWR model are not randomly distributed but instead show significant
spatial autocorrelation. The graph shows that the majority of the
residuals are close to zero, which means the predicted values are close
to the observed values. This indicates that the GWR model has accounted
for spatial variation effectively and that it is a good fit.

This methodological outline ensures that each step of the data handling,
analysis, and visualization process is documented, providing
transparency and reproducibility of the research findings.

\begin{figure}[!ht]

\centering{

\includegraphics{report_files/figure-pdf/fig-1-1.pdf}

}

\caption{\label{fig-1}Higher death rates can be seen in Stroke Belt
region for 2015}

\end{figure}%

\section{Results}\label{results}

\subsection{Regression Summary}\label{regression-summary}

We used two models: global regression and GWR regression output. They
show the relationships between socioeconomic, demographic, and
environmental variables and death rates. The first is a global
regression model that does not consider spatial correlation despite
revealing that all predictors are statistically significant. Hence,
while the model does suggest that our variables are indeed important,
the global model may overlook local variations that are crucial in
understanding the true nature of the data. This can be seen in
Figure~\ref{fig-2}, which shows the residual of the global model, and
there appears to be a spatial pattern to the residuals, with specific
areas showing clusters of higher residuals and the other regions showing
clusters of lower residuals. This clustering of residuals suggests that
the global regression model may not be capturing all the spatial
variation in the data. This implies that the relationship between the
independent and dependent variables might differ across different
locations.

On the other hand, the geographically weighted regression (GWR) model
incorporates spatial variation, which is a critical factor given the
data context. In summary, by accommodating the spatial component present
in the data, the GWR model provides a more realistic interpretation of
how various factors influence death rates across different regions.

\begin{figure}

\centering{

\captionsetup{labelsep=none}\includegraphics{report_files/figure-pdf/fig-2-1.pdf}

}

\caption{\label{fig-2}}

\end{figure}%

\subsection{Local Significance Plots}\label{local-significance-plots}

In Figure~\ref{fig-1}, the plot shows an exploratory data analysis (EDA)
plot, specifically a choropleth map displaying the number of deaths from
cardiovascular disease (CVD) per 100,000 people across the contiguous
United States for the year 2015. The regions along the higher latitudes
(nearing 45 N) and towards the eastern section (approaching 80 W)
display darker shades, suggesting higher CVD death rates in these areas,
specifically the midwest and southwest regions.~

Figure~\ref{fig-3} represents the local significance and magnitude of
the `\% White' demographic parameter on cardiovascular disease outcomes.
We plotted for the local significance and magnitude because it can help
identify areas where the predictor variable has a stronger or weaker
influence on the outcome, leading to targeted insights that would not be
possible with a global model. From the plot, the prominent dark red
areas in the central part of the country, extending towards the
southeastern regions, indicate a significant negative correlation
between the percentage of the white population and CVD outcomes in these
areas. Like the Northwest region, however, there are few areas,
particularly in the southwest near New Mexico and Arizona, where there
are positively correlated to CVD death rate, with the white regions of
counties suggesting no significance and this is white region is similar
for all the figures.

Figure~\ref{fig-4} represents the local significance and magnitude of
the `\% African' demographic parameter on cardiovascular disease
outcomes. From the plot, areas located approximately in the northern
central region indicate a significant positive correlation to CVD rate,
and areas, notably in the central to southeastern areas of the map,
suggest a significant negative correlation.

Figure~\ref{fig-5} represents the local significance and magnitude of
the `\% Hispanic' demographic parameter on cardiovascular disease
outcomes. From the plot, areas across the central and southeastern
regions are shaded in red, suggesting a significant negative correlation
between the percentage of the Hispanic population and CVD outcomes. The
isolated red patches in the north-central region indicate areas where an
increased Hispanic population correlates with higher CVD outcomes.~

Figure~\ref{fig-6} represents the local significance and magnitude of
the `\% Asian' demographic parameter on cardiovascular disease outcomes.
From the plot, a substantial portion of the map, particularly across the
central to eastern regions, is colored in various shades of red. This
suggests that in these areas, an increased percentage of the Asian
population correlates with lower CVD outcomes. However, there are some
parts of the West where higher percentages of the Asian population are
associated with an increase in CVD outcomes.~

Figure~\ref{fig-7} represents the local significance and magnitude of
the ``\%p2.5 air quality'' parameter on cardiovascular disease outcomes.
From the plot, in the southern and northeastern regions, there increased
levels of PM2.5 colored as red in the plot, meaning they are associated
with higher rates of CVD. There are some regions in the West where it
was negatively correlated with CVD, highlighted in blue.

Figure~\ref{fig-8} represents the local significance and magnitude of
the ``median income'' parameter on cardiovascular disease outcomes. From
the plot, most of the region suggests a significant negative correlation
between the median income parameter and CVD outcomes. However, some
parts of Texas indicate a positive correlation to CVD outcomes.~

Figure~\ref{fig-9} represents the local significance and magnitude of
the ``Unemployment'' parameter on cardiovascular disease outcomes. From
the plot, there is a positive correlation to CVD rates in most of the
central region, and in the southwest region, there is a negative
correlation.

\begin{figure}

\begin{minipage}{0.50\linewidth}

\centering{

\includegraphics{PresentationPhotos/whiteplot.png}

}

\subcaption{\label{fig-3}Figure 3a: Strong effect in southern New
England area}

\end{minipage}%
%
\begin{minipage}{0.50\linewidth}

\centering{

\includegraphics{PresentationPhotos/blackplot.png}

}

\subcaption{\label{fig-4}Figure 3b: Significant positive effect in Idaho
region}

\end{minipage}%
\newline
\begin{minipage}{0.50\linewidth}

\centering{

\includegraphics{PresentationPhotos/hispanplot.png}

}

\subcaption{\label{fig-5}Figure 3c: Positive effect focused in New
England}

\end{minipage}%
%
\begin{minipage}{0.50\linewidth}

\centering{

\includegraphics{PresentationPhotos/asianplot.png}

}

\subcaption{\label{fig-6}Figure 3d: Strong positive effects focused in
western Stroke Belt, minor positive effects through region}

\end{minipage}%

\end{figure}%

\begin{figure}

\centering{

\includegraphics{report_files/figure-pdf/fig-7-1.pdf}

}

\caption{\label{fig-7}Higher effect in south and northeast. Negative
effect in the west.}

\end{figure}%

\begin{figure}

\begin{minipage}{0.50\linewidth}

\centering{

\includegraphics{PresentationPhotos/medIncomePlot.png}

}

\subcaption{\label{fig-8}Figure 5a: As expected, median income has a
consistent negative relationship with CVM}

\end{minipage}%
%
\begin{minipage}{0.50\linewidth}

\centering{

\includegraphics{PresentationPhotos/unemployPlot.png}

}

\subcaption{\label{fig-9}Figure 5b: Consistently positive relationship
in central regions, however negative patches exist in Florida and
northern states}

\end{minipage}%

\end{figure}%

\section{Discussion}\label{discussion}

The purpose of this study was to fill a gap in prior studies where the
impacts of CVD were primarily studied within the Stroke Belt, rather
than the country at large.

Our goal was to put our findings towards answering the following
question: where are the socioeconomic and environmental factors
affecting CVD rates in the United States?

These maps reveal that the relationships between race, socio-economic
factors, environmental quality, and death rates are complex and highly
localized. The significance and strength of these relationships vary
considerably across different parts of the United States. In contrast,
some of the socio-economic factors such as median income show
widespread, consistent significance, implying that the significance of
the relationship with CVD outcomes is nearly constant across various
locations. This differs from the heterogeneous local significance that
we observed in the majority of our other variables.

Another factor to consider from the plots is the percentage of each race
that inhabits each county. We can observe that for African Americans,
Hispanics, and Asians that the impacts are significant in localized
areas of the country, which may reflect underlying health disparities in
access to medical care. One limitation of race percentages is the
misreporting of medical records affecting minorities (Tabb et al. 2020).
However, this oversight lends further credence to the fact that
intervention is needed in order to combat racial health disparities. We
can look at the Variance Inflation Factor (VIF) to determine if
multi-collinearity has an impact on our results:

\begin{table}

\caption{\label{tbl-10}}

\centering{

\captionsetup{labelsep=none}

\begin{tabular}{rrrrrrr}
  \hline
\%White & \%Black & \%Hispanic & \%Asian & PM2.5 & MedIncome & Unemployment \\ 
  \hline
183.00 & 196.00 & 7.00 & 3.00 & 1.00 & 2.00 & 3.00 \\ 
  81.00 & 94.00 & 4.00 & 3.00 & 2.00 & 3.00 & 3.00 \\ 
  302.00 & 314.00 & 17.00 & 4.00 & 1.00 & 2.00 & 2.00 \\ 
  141.00 & 151.00 & 5.00 & 2.00 & 1.00 & 2.00 & 3.00 \\ 
  132.00 & 128.00 & 8.00 & 3.00 & 1.00 & 2.00 & 2.00 \\ 
  283.00 & 294.00 & 14.00 & 4.00 & 1.00 & 2.00 & 2.00 \\ 
   \hline
\end{tabular}

}

\end{table}%

Typically, when examining the VIF of a GWR model, we want the
coefficients for each of the seven columns--which represent our seven
independent variables--to be less than 15. We can see in
Table~\ref{tbl-10} that this standard holds for most cases of the
columns outside the \% White column, which has extremely high values.
This represents a limitation in our model, as it shows multicollinearity
with respect to the white and black variables, however, this is to be
expected given that the racial variables we are using sum up to one. An
adjustment to the model could be made in a future study to eliminate the
\% White variable and aggregate the other three into a single \%
Non-White variable. .

The concentration of PM2.5 and the consequential reduction in air
quality has a significant impact localized in the central and
southeastern regions of the United States, suggesting environmental
health concerns that might require region-specific intervention. This
aligns with previous research done on the stroke belt, where factors
such as smoking, limited access to healthcare, and food insecurity
(Zelko et al. 2023) combined with the high concentrations of PM2.5 have
created a hotspot of CVM far surpassing the rates of the rest of the
country.

We have shown that by using a GWR model to analyze the relationship
between CVM and socioeconomic covariates, the factors that have the most
significant impact on death rates vary by area of the country. This
highlights the need to identify methods of intervention in order to curb
one of the deadliest groups of diseases on the planet (Marlow 1994).

\section{Appendix}\label{appendix}

The R code used for this study can be found in our public GitHub
repository: \url{https://github.com/jbooc117/STAT489-Project.git}.

\subsection{Simulation Study Figures}\label{simulation-study-figures}

\begin{longtable}[]{@{}lll@{}}
\caption{Moran's I test demonstrates spatial correlation in our
model}\tabularnewline
\toprule\noalign{}
Moran I Statistic & Expectation & Variance \\
\midrule\noalign{}
\endfirsthead
\toprule\noalign{}
Moran I Statistic & Expectation & Variance \\
\midrule\noalign{}
\endhead
\bottomrule\noalign{}
\endlastfoot
0.3010445156 & 0.0003274394 & 0.0001178233 \\
\end{longtable}

\begin{figure}

\centering{

\includegraphics{PresentationPhotos/simplot.png}

}

\caption{\label{fig-A1}Division of country into four quadrants shows the
effectiveness of the GWR model}

\end{figure}%

\newpage{}

\section*{References}\label{references}
\addcontentsline{toc}{section}{References}

\phantomsection\label{refs}
\begin{CSLReferences}{1}{0}
\bibitem[\citeproctext]{ref-gebreabExploringRacialDisparities2012}
Gebreab, Samson Y., and Ana V. Diez Roux. 2012. {``Exploring Racial
Disparities in {CHD} Mortality Between Blacks and Whites Across the
{United} {States}: {A} Geographically Weighted Regression Approach.''}
\emph{Health \& Place} 18 (5): 1006--14.
\url{https://doi.org/10.1016/j.healthplace.2012.06.006}.

\bibitem[\citeproctext]{ref-GWmodel}
Gollini, Isabella, Binbin Lu, Christopher Brunsdon, and Paul Harris.
2015. {``{\textbraceleft}GWmodel{\textbraceright}: An
{\textbraceleft}r{\textbraceright} Package for Exploring Spatial
Heterogeneity Using Geographically Weighted Models''} 63.
\url{https://doi.org/10.18637/jss.v063.i17}.

\bibitem[\citeproctext]{ref-liuAnalysisShortTermEffects2020}
Liu, Yi, Jingjie Sun, Yannong Gou, Xiubin Sun, Dandan Zhang, and Fuzhong
Xue. 2020. {``Analysis of {Short}-{Term} {Effects} of {Air} {Pollution}
on {Cardiovascular} {Disease} {Using} {Bayesian} {Spatio}-{Temporal}
{Models}.''} \emph{International Journal of Environmental Research and
Public Health} 17 (3): 879.
\url{https://doi.org/10.3390/ijerph17030879}.

\bibitem[\citeproctext]{ref-Marlow1994}
Marlow, Hilary F. 1994. {``The Pharmaceutical Industry Viewpoint.''}
\emph{Cardiology} 85 (1): 102--12.
\url{https://doi.org/10.1159/000176769}.

\bibitem[\citeproctext]{ref-singhSpatiotemporalDemographicTrends2019}
Singh, Gitanjali M., Ninon Becquart, Melissa Cruz, Andrea Acevedo,
Dariush Mozaffarian, and Elena N. Naumova. 2019. {``Spatiotemporal and
{Demographic} {Trends} and {Disparities} in {Cardiovascular} {Disease}
{Among} {Older} {Adults} in the {United} {States} {Based} on 181
{Million} {Hospitalization} {Records}.''} \emph{Journal of the American
Heart Association} 8 (21): e012727.
\url{https://doi.org/10.1161/JAHA.119.012727}.

\bibitem[\citeproctext]{ref-tabbExploringSpatialPatterning2020}
Tabb, Loni Philip, Angel Ortiz, Suzanne Judd, Mary Cushman, and Leslie
A. McClure. 2020. {``Exploring the {Spatial} {Patterning} in {Racial}
{Differences} in {Cardiovascular} {Health} {Between} {Blacks} and
{Whites} {Across} the {United} {States}: {The} {REGARDS} {Study}.''}
\emph{Journal of the American Heart Association} 9 (9): e016556.
\url{https://doi.org/10.1161/JAHA.120.016556}.

\bibitem[\citeproctext]{ref-terryTrendsCardiovascularDisease2023}
Terry, Katrina, Mohamed Makhlouf, Salah E. Altarabsheh, Vaishali Deo,
Fanny Petermann-Rocha, Yakov Elgudin, Khurram Nasir, Sanjay Rajagopalan,
Sadeer Al-Kindi, and Salil Deo. 2023. {``Trends in {Cardiovascular}
{Disease} {Mortality} by {County}-{Level} {Social} {Vulnerability}
{Index} in the {United} {States}.''} \emph{Journal of the American Heart
Association} 12 (20): e030290.
\url{https://doi.org/10.1161/JAHA.123.030290}.

\bibitem[\citeproctext]{ref-warsitoRobustGeographicallyWeighted2018}
Warsito, Budi, Hasbi Yasin, Dwi Ispriyanti, and Abdul Hoyyi. 2018.
{``Robust Geographically Weighted Regression of Modeling the {Air}
{Polluter} {Standard} {Index} ({APSI}).''} \emph{Journal of Physics:
Conference Series} 1025 (May): 012096.
\url{https://doi.org/10.1088/1742-6596/1025/1/012096}.

\bibitem[\citeproctext]{ref-zelkoGeographicallyWeightedModeling2023}
Zelko, Andrea, Pedro R. V. O. Salerno, Sadeer Al-Kindi, Fredrick Ho,
Fanny Petermann Rocha, Khurram Nasir, Sanjay Rajagopalan, Salil Deo, and
Naveed Sattar. 2023. {``Geographically {Weighted} {Modeling} to
{Explore} {Social} and {Environmental} {Factors} {Affecting}
{County}-{Level} {Cardiovascular} {Mortality} in {People} {With}
{Diabetes} in the {United} {States}: {A} {Cross}-{Sectional}
{Analysis}.''} \emph{The American Journal of Cardiology} 209 (December):
193--98. \url{https://doi.org/10.1016/j.amjcard.2023.09.084}.

\end{CSLReferences}



\end{document}
